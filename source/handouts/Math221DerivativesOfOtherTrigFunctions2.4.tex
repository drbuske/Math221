\documentclass[11pt,a4paper]{report}

\usepackage{amssymb,amsmath,epsfig,float,subfig,hyperref,multicol}

\usepackage{xcolor}

\definecolor{SCSUred}{HTML}{CD1041}

\hypersetup{colorlinks=true,linkcolor=SCSUred,urlcolor=SCSUred}

\usepackage{enumerate}
\usepackage{tikz}
\usetikzlibrary{arrows}
\usetikzlibrary{patterns}
\usetikzlibrary{decorations}
%%\usetikzlibrary{intersections}
\usetikzlibrary{matrix}
\usetikzlibrary{snakes}
\usetikzlibrary{calc}
\usetikzlibrary{backgrounds}

\definecolor{linecolor}{HTML}{0074C8}
\definecolor{linecolor2}{HTML}{C80200}

\newcommand{\imagebullet}[1]{\includegraphics[width=0.5cm]{#1}}

\pagestyle{empty}
\setlength{\textwidth}{7in}
\setlength{\textheight}{10in}
\setlength{\oddsidemargin}{-25pt}
\setlength{\evensidemargin}{-25pt}
\setlength{\topmargin}{-50pt}

\usepackage[english]{babel}
\usepackage[utf8]{inputenc}
\usepackage{fancyhdr}
 
%%\pagestyle{fancy}
\renewcommand{\headrulewidth}{0pt}
%%\fancyhf{}
%%\rhead{Share\LaTeX}
%%\lhead{Guides and tutorials}
%%\cfoot{OVER}

\newcommand{\DueA}{Friday, October 4}
\newcommand{\DueB}{Friday, October 4}
\newcommand{\DueC}{Monday, October 14}

\begin{document}


\begin{figure}[ht]
\begin{flushright}
	\includegraphics[width=2.0in]{U_PriHorz_WhtLtBG.jpg}
	\end{flushright}
\end{figure}

\vspace{-12mm}

\begin{flushleft}
\Large\bf \href{https://activecalculus.org/single/sec-2-4-other-trig.html}{2.4 - Derivatives of Other Trigonometric Functions}\rm
%%Daily Preparation - \DueA \rm
\end{flushleft}


\vspace{8pt}

\noindent {\Large\bf{Overview}} \\
This section covers the following concepts: Derivatives of $\tan(x), \cot(x), \sec(x)$, and $\csc(x)$.

%%This section covers the following concept: Linearization of a function.



\vspace{16pt}

%%\pagebreak

\noindent {\Large\bf{To prepare for class}} \\
Complete all actions listed below.  Respond to the questions highlighted with {\color{SCSUred}{\boxed{Submit}}}.  %% by the start of class on {\bf{\DueA}}.  A single .pdf should be uploaded to D2L Brightspace. 
\begin{itemize} \itemsep -2pt % Reduce space between items

\item[{\color{SCSUred} \boxed{Submit}}]  {\bf{Explore}} the applet \href{https://www.geogebra.org/m/cNEtsbvC}{Sine, Cosine, and Tangent animated from the unit circle}.  Before starting this section, it is important to remember how the tangent function is related to the sine and cosine functions.  Use this applet to answer the following questions.  Submit screen captures and answers to the questions posed as needed. 
\begin{itemize}
\item Check the $\sin \theta$ box.  As you move the slider for the angle $\theta$, the point $P$ on the unit circle moves and the graph of $\sin \theta$ is plotted.  How is the height of the blue point on the graph above $\theta$ related to point $P$?  Hint: Look at the {\textcolor{blue}{blue}} dashed line in the unit circle.  
\item Now check the $\cos \theta$ box.  As you move the slider for the angle $\theta$, the point $P$ on the unit circle moves and the graph of $\cos \theta$ is plotted.  How is the height of the red point on the graph above $\theta$ related to point $P$?  Hint: Look at the {\textcolor{red}{red}} dashed line in the unit circle. 
\item Now check the $\tan \theta$ box.  As you move the slider for the angle $\theta$, the point $P$ on the unit circle moves and the graph of $\tan \theta$ is plotted.  How is the height of the purple point on the graph above $\theta$ related to point $P$?  Hint: Look at the {\textcolor{red}{red}} and {\textcolor{blue}{blue}} dashed lines in the unit circle. 
\item What fundamental trigonometric identity involving $\sin \theta, \cos \theta$, and $\tan \theta$ does this exercise reinforce?  
\end{itemize}

\item {\bf{Explore}} the applet located at \href{https://www.geogebra.org/m/hFReSaqJ}{https://www.geogebra.org/m/hFReSaqJ}.  This applet gives you a geometric way to think about the value of $\tan \beta$ when $0^{\circ} \leq \beta \leq 90^{\circ}$.  As you change the angle (via the slider), it strongly suggests that the length of $\overline{TB}$ is $\tan \beta$.  Knowing that $\tan \beta$ is the ratio of lengths of $\overline{DP}$ to $\overline{OD}$ where $O$ is the origin, how could we prove that indeed the length of $\overline{TB}$ is equal to $\tan \beta$?  

\item {\bf{Read}} motivating questions and the introduction to \href{https://activecalculus.org/single/sec-2-4-other-trig.html}{section 2.4} (up until Preview Activity 2.4.1).
\item[{\color{SCSUred} \boxed{Submit}}] {\bf{Do}} \href{https://activecalculus.org/single/sec-2-4-other-trig.html#hHS}{Preview Activity 2.4.1}.
\item[{\color{SCSUred} \boxed{Submit}}] {\bf{Do}} the following extension of \href{https://activecalculus.org/single/sec-2-4-other-trig.html#hHS}{Preview Activity 2.4.1}.  Submit a screen capture from {\it{GeoGebra}} and answer questions posed.  
\begin{itemize}
\item In \href{https://www.geogebra.org/classic}{\it{GeoGebra}}, type {\tt{tan(x)}} in the input box.  A graph will appear.  
\item Based on the graph, $f(x)=\tan (x)$ is periodic.  What is the period?  Do you expect the derivative $f'(x)$ to have the same period?  Why or why not?  
\item Is the slope of the graph of $f(x)=\tan (x)$ always positive?  always negative?  What does this suggest about the graph of the derivative?
\item Based on the graph of $f(x) = \tan (x)$, on a single period $(-\pi,\pi)$, what can you say about concavity?  What does this suggest about the graph of $f'(x)$ on $(-\pi,\pi)$?  
\item In \href{https://www.geogebra.org/classic}{\it{GeoGebra}}, type {\tt{Derivative(f(x))}} in the next input box.  The graph of $f'(x)$ will then appear.  Do the properties you expected above all appear?  
\end{itemize}  

\item {\bf{Read}} \href{https://activecalculus.org/single/sec-2-4-other-trig.html#eiw}{section 2.4.1} which discusses the derivative of $\cot x$.

\item {\bf{Watch}} video \href{https://www.youtube.com/watch?v=wARt0oF46wg&feature=emb_title}{Quick Review - Derivatives of Other Trig Functions (1:55)}.  

\item {\bf{Watch}} video \href{https://www.youtube.com/watch?v=43UXLvQgmwY&feature=emb_title}{Examples of Other Trig Derivatives (9:46)}.

\item {\bf{Do}}  \href{https://activecalculus.org/single/sec-2-4-other-trig.html#ZgR}{Exercises 1-5 in section 2.4}.

 



 

\end{itemize}










\vspace{16pt}

\noindent {\Large\bf{After class}}\\
Solidifying the concepts discussed in class through practice is necessary to build your skills. 

%%\noindent {\large\bf{After \DueA}}
\begin{itemize}\itemsep -2pt % Reduce space between items

\item {\bf{Do}} \href{https://activecalculus.org/single/sec-2-4-other-trig.html#Pwg}{Activity 2.4.2} and \href{https://activecalculus.org/single/sec-2-4-other-trig.html#lOe}{Activity 2.4.3}.

\item[\imagebullet{CopilotLogo.jpg}] Prompt {\bf{Copilot}} ``I want a pneumonic device to remember the derivatives of the 6 basic trigonometric functions."  Is the device it returns useful to you?  Or more confusing?  
 
\item[\imagebullet{CopilotLogo.jpg}] Prompt {\bf{Copilot}} ``How does the quotient rule in calculus give the derivative of $\cot(x)$?"  Is the response a correct one?  
  
\item {\bf{Read}} \href{https://activecalculus.org/single/sec-2-4-other-trig.html#KpF}{section 2.4.2 (Summary)}.  
  
\item {\bf{Do}}  \href{https://activecalculus.org/single/sec-2-4-other-trig.html#lgO}{Exercises 6-8 in section 2.4}. 
 


%%\end{itemize}

%%\noindent {\large\bf{After Monday, October 5}}
%%\begin{itemize}\itemsep -2pt % Reduce space between items
%%\item {\bf{Finish}} any in-class activities you might not have finished during class.
  
\item {\bf{Start working}} on the \href{https://www.myopenmath.com/index.php}{MOMwork} (MyOpenMath) assignment for this section.  %%This will be due on \DueC.  
\end{itemize}

%%\pagebreak

\vspace{16pt}

\noindent {\Huge\bf{Extra Prep}}

 \vspace{16pt}

\noindent {\Large\bf{Basic learning objectives}}\\
These are the tasks you should be able to perform with reasonable
fluency when you arrive at our next class meeting. Important new
vocabulary words are indicated {\it{in italics}}.  Check each box when you feel confident you have a firm grasp on that objective.  

\begin{itemize} \itemsep -2pt % Reduce space between items
\renewcommand{\labelitemi}{\scriptsize$\square$}
\item State the basic properties of all six main trigonometric functions: definition, domain, graph, values of the function at the main angles ($0,\pi/6,\pi/4,\pi/3,\pi/2$ and all geometrically related angles), related trigonometric identities.
\item State the derivatives of $\tan(x), \cot(x), \sec(x)$, and $\csc(x)$.
\end{itemize}

\vspace{16pt}

\noindent {\Large\bf{Advanced learning objectives}}\\
In addition to mastering the basic objectives, here are the tasks you should be able to perform after class, with practice: 
\begin{itemize} \itemsep -2pt % Reduce space between items
\renewcommand{\labelitemi}{\scriptsize$\square$}
\item Derive the derivatives of $y=\tan x, y=\sec x, y=\cot x, and y=\csc x$ ``from scratch'' using only basic derivative rules and trigonometric identities.

\item Differentiate a function for which the derivative involves a combination of these trigonometric functions, and other rules we've learned.

\item Use all rules learned so far in the context of a real-world problem to find the slope of a tangent line, the instantaneous rate of change in a function, or the instantaneous velocity of an object.

\end{itemize}

\vspace{16pt}

\noindent {\Large\bf{Need More Help?}}

\begin{itemize}\itemsep -2pt % Reduce space between items
\item {\bf{Watch}} video: \href{https://www.youtube.com/watch?v=tyjM1j59fiw}{Remembering Trig Derivatives (3:21)}.  This is intended to help you easily remember these six basic rules. 
\end{itemize}

\end{document}

