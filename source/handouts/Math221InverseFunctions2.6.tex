\documentclass[11pt,a4paper]{report}

\usepackage{amssymb,amsmath,epsfig,float,subfig,hyperref,multicol}

\usepackage{xcolor}

\definecolor{SCSUred}{HTML}{CD1041}

\hypersetup{colorlinks=true,linkcolor=SCSUred,urlcolor=SCSUred}

\usepackage{enumerate}
\usepackage{tikz}
\usetikzlibrary{arrows}
\usetikzlibrary{patterns}
\usetikzlibrary{decorations}
%%\usetikzlibrary{intersections}
\usetikzlibrary{matrix}
\usetikzlibrary{snakes}
\usetikzlibrary{calc}
\usetikzlibrary{backgrounds}

\definecolor{linecolor}{HTML}{0074C8}
\definecolor{linecolor2}{HTML}{C80200}

\newcommand{\imagebullet}[1]{\includegraphics[width=0.5cm]{#1}}


\pagestyle{empty}
\setlength{\textwidth}{7in}
\setlength{\textheight}{10in}
\setlength{\oddsidemargin}{-25pt}
\setlength{\evensidemargin}{-25pt}
\setlength{\topmargin}{-50pt}

\usepackage[english]{babel}
\usepackage[utf8]{inputenc}
\usepackage{fancyhdr}
 
%%\pagestyle{fancy}
\renewcommand{\headrulewidth}{0pt}
%%\fancyhf{}
%%\rhead{Share\LaTeX}
%%\lhead{Guides and tutorials}
%%\cfoot{OVER}

\newcommand{\DueA}{Thursday, October 10}
\newcommand{\DueB}{Friday, October 11}
\newcommand{\DueC}{Friday, October 18}

\begin{document}


\begin{figure}[ht]
\begin{flushright}
	\includegraphics[width=2.0in]{U_PriHorz_WhtLtBG.jpg}
	\end{flushright}
\end{figure}

\vspace{-12mm}

\begin{flushleft}
\Large\bf \href{https://activecalculus.org/single/sec-2-6-inverse.html}{2.6 - Derivatives of Inverse Functions}\rm
%%Daily Preparation - \DueA \rm
\end{flushleft}


\vspace{8pt}

\noindent {\Large\bf{Overview}} \\
Functions are a powerful tool in mathematics: each describes a rule or process that takes any valid input to one and only one output. One of the natural questions that arises regarding any function is ``can the rule be undone (or reversed)?" Indeed, this is connected to the concept of a function's inverse. While not every function has an inverse, for those that do, knowing the inverse can be valuable in a wide range of settings.\\
\\
\noindent
Among the most important inverse functions in mathematics are the natural logarithm function, $\ln(x)$ (which is the inverse of the exponential function, $e^x$) and the inverse trigonometric functions, such as $\arcsin(x)$ and $\arctan(x)$.\\
\\
\noindent
In this section of the text, we explore how we can use the relationship between a function and its inverse to determine the derivative formula for the inverse function and, along the way, learn a key geometric connection between a function's derivative and the derivative of its inverse.



\vspace{16pt}

%%\pagebreak

\noindent {\Large\bf{To prepare for class}} \\
Complete all actions listed below.  Respond to the questions highlighted with {\color{SCSUred}{\boxed{Submit}}}.  %% by the start of class on {\bf{\DueA}}.  A single .pdf should be uploaded to D2L Brightspace. 
\begin{itemize} \itemsep -2pt % Reduce space between items


\item {\bf{Read}} motivating questions and the introduction to \href{https://activecalculus.org/single/sec-2-6-inverse.html}{section 2.6} (up until Preview Activity 2.6.1).
\item[{\color{SCSUred} \boxed{Submit}}]  {\bf{Do}} \href{https://activecalculus.org/single/sec-2-6-inverse.html#xAK}{Preview Activity 2.6.1}.
\begin{itemize}
\item (Optional) {\bf{Watch}} video \href{https://www.youtube.com/watch?v=pmtCt2gmmws}{solution to Preview Activity 2.6.1 (3:45)}
\end{itemize}
\item {\bf{Read}} \href{https://activecalculus.org/single/sec-2-6-inverse.html#OSu}{section 2.6.1}.

\item[{\color{SCSUred} \boxed{Submit}}] {\bf{Do}} the following to construct the function $\log_2(x)$ (the inverse of the exponential function $2^x$) in \href{https://www.geogebra.org/classic?lang=en}{GeoGebra}.  Submit screenshots as needed.
\begin{itemize}
\item Type {\tt{2$\wedge$x}} in the first input box.  {\it{GeoGebra}} will label this as the function $f(x)$ and will sketch its plot.  
\item We now sketch a diagonal line about which we will reflect.  The third box on the menu offers the option {\tt{Lines}}.  Choose it and then point to the origin $(0,0)$ and click to place point $A$ and then point to $(1,1)$ and click to place point $B$.  A line will be constructed through these two points. Remember that two points determine a line.  [Yes, this is the line $y=x$.]  
\begin{figure}[H]
\begin{center}
	\includegraphics[width=3.0in]{LinesGeoGebra.jpg}
	\end{center}
\end{figure}
\item We are now ready to reflect the graph of $f(x)$ about this line. {\it{GeoGebra}} has a built-in tool to do this.  Select the ninth box on the menu and the option {\tt{Reflect about line}}.  Choose it.  Then, click on the graph of $2^x$ to select it.  Second, click on the line $y=x$ to select it.  The reflection of the graph of $y=2^x$ should appear.  
\begin{figure}[H]
\begin{center}
	\includegraphics[width=3.0in]{ReflectGeoGebra.jpg}
	\end{center}
\end{figure}
\item To complete our replication of Figure 2.6.2 in the text, let's plot the point $\displaystyle (-1,\frac{1}{2})$.  In the next input box, type {\tt{C = (-1,f(-1))}}.  The point $\displaystyle (-1,\frac{1}{2})$ will be plotted on your graph of $f(x)$.  Repeat the process used to reflect the graph about the line $y=x$ to reflect the {\it{point}} $C$ about this same line: From the menu, select {\tt{Reflect about line}}, then select point {\tt{C}} on the graph and finally select the line $y=x$.  A new point on the graph of the inverse of $2^x$ should appear at $\displaystyle (\frac{1}{2}, -1)$. 
\item To verify that you have indeed found the graph of $\log_2(x)$, type {\tt{log2(x)}} in the next input box.  {\it{GeoGebra}} will plot this function right on top of your reflection if you have followed this procedure correctly.  
\item Let's try this for a second function.  Delete the graph of $\log_2(x)$.  Then, go to the first input box and change {\tt{2$\wedge$x}} to {\tt{sqrt(x)}}.  Your output should be automatically updated.  Does it look correct?  How do you know?  
\end{itemize}

\item {\bf{Do}} the following problem.  Use the graph of $2^x$ and the graph of $\log_2(x)$ found during your {\it{GeoGebra}} construction if you wish.  [Note: The co-domain is defined in the reading.]  
\begin{enumerate}
\item
\begin{enumerate}
\item $2^{\log_2(x)} = \underline{\hspace{20mm}}$
\item $\log_2(2^x) = \underline{\hspace{20mm}}$
\item The domain of $f(x) = 2^x$ is $\underline{\hspace{20mm}}$.  
The co-domain is $\underline{\hspace{20mm}}$.  

\item The domain of $f^{-1}(x) = \log_2(x)$ is $\underline{\hspace{20mm}}$.  
The co-domain is $\underline{\hspace{20mm}}$.
\item $y= \log_2(x)$ can be rewritten as $x = \underline{\hspace{30mm}}$.
\end{enumerate} 
\end{enumerate}

\item {\bf{Read}} \href{https://activecalculus.org/single/sec-2-6-inverse.html#uZD}{section 2.6.2}.

\item {\bf{Watch}} video \href{https://www.youtube.com/watch?v=chdkxtt8XQo&list=PL9bIjQJDwfGuXQHuS5Jkmum_CFILoCZX-&index=46}{Quick Review - Derivatives of Inverse Functions (3:24)}.  


\item[{\color{SCSUred} \boxed{Submit}}]  {\bf{Watch}} video \href{https://www.youtube.com/watch?v=jhBhSerqbyU&feature=emb_title}{Examples of Derivatives with the Natural Log (6:29)}.  In this video, the chain rule is used to take a derivative of a composite function of the form $f(g(x))$.  Identify $g(x)$, $f(x)$, $g'(x)$ and $f'(x)$ for that example. 

%%\item {\bf{Do}} Activity 2.6.2.  

\item[\imagebullet{CopilotLogo.jpg}] Prompt {\bf{Copilot}} ``How can one compute the derivative of $\ln(x)$ using the fact that the derivative of exp(x) is exp(x)?" 

\item {\bf{Do}}  \href{https://activecalculus.org/single/sec-2-6-inverse.html#YSJ}{Exercises 1-2 in section 2.6.6}.

\end{itemize}











\vspace{16pt}

\noindent {\Large\bf{After class}}\\
Solidifying the concepts discussed in class through practice is necessary to build your skills. 

%%\noindent {\large\bf{After \DueA}}
\begin{itemize}\itemsep -2pt % Reduce space between items
\item {\bf{Read}} \href{https://activecalculus.org/single/sec-2-6-inverse.html#bgM}{section 2.6.3}.
\item {\bf{Watch}} video \href{https://www.youtube.com/watch?v=pEEQNdttZsw&feature=emb_title}{Derivatives Involving $\arcsin(x)$ (5:26)}.  

%%\item {\bf{Do}} Activity 2.6.3. 

\item[\imagebullet{CopilotLogo.jpg}] Prompt {\bf{Copilot}} ``How is the formula for the derivative of $\arctan(x)$ derived?"

\item {\bf{Read}} \href{https://activecalculus.org/single/sec-2-6-inverse.html#HnV}{section 2.6.4}. 



\item {\bf{Explore}} an applet \href{http://webspace.ship.edu/msrenault/GeoGebraCalculus/derivative_inverse_functions.html}{Derivatives of Inverse Functions} to better understand the steps involved in the process of finding the derivative of an inverse function.  Be sure to read the {\it{Explore}} questions for each step on the slider.  

\item {\bf{Do}}  \href{https://activecalculus.org/single/sec-2-6-inverse.html#OAI}{Exercises 3-4 in section 2.6}.
  


 


%%\end{itemize}

%%\noindent {\large\bf{After \DueB}}
%%\begin{itemize}\itemsep -2pt % Reduce space between items 

\item {\bf{Read}} \href{https://activecalculus.org/single/sec-2-6-inverse.html#nve}{section 2.6.5 (summary)}.  

\item {\bf{Do}}  \href{https://activecalculus.org/single/sec-2-6-inverse.html#aPa}{Exercises 5-8 in section 2.6.6}.

\item {\bf{Do}}  \href{https://activecalculus.org/single/sec-2-6-inverse.html#zrK}{Exercises 9-12 from section 2.6.6}.

\item {\bf{Start working}} on the \href{https://www.myopenmath.com/index.php}{MOMwork} (MyOpenMath) assignment for this section.  %%This will be due on \DueC. 

\end{itemize}

\pagebreak

\vspace{16pt}

\noindent {\Huge\bf{Extra Prep}}

\vspace{16pt}

\noindent {\Large\bf{Basic learning objectives}}\\
These are the tasks you should be able to perform with reasonable
fluency when you arrive at our next class meeting. Important new
vocabulary words are indicated {\it{in italics}}.  Check each box when you feel confident you have a firm grasp on that objective. 

\begin{itemize} \itemsep -2pt % Reduce space between items
\renewcommand{\labelitemi}{\scriptsize$\square$}
\item State the definition of an inverse function.
%%\item Apply the horizontal line test and explain what it tells us about a function.

\item Recognize that writing $y=f(x)$ and $x=f^{-1}(y)$ say the exact same thing. 
\item Illustrate the geometric relationship between $y=e^x$ and $y=\ln(x)$ and between $y=f(x)$ and $y=f^{-1}(x)$ in general.
\item Calculate basic values of the natural logarithm, arcsine, and arctangent functions without a calculator (e.g. $\ln(e^5)$ and $\arcsin(1/2)$).  
\end{itemize}

\vspace{16pt}

\noindent {\Large\bf{Advanced learning objectives}}\\
In addition to mastering the basic objectives, here are the tasks you should be able to perform after class, with practice:
\begin{itemize} \itemsep -2pt % Reduce space between items
\renewcommand{\labelitemi}{\scriptsize$\square$}
\item Explain how the derivative of an inverse function is related to the derivative of the original function.
\item Use derivative rules for $\ln(x)$, $\arcsin(x)$, and $\arctan(x)$.  
\item Use the relationship between a function and its inverse to develop the inverse function's derivative rule (in particular: be able to do this ``from scratch" for $\ln(x)$, $\arcsin(x)$, and $\arctan(x)$).  
\item Differentiate a function involving logarithmic functions, arcsine and arctangent functions, and for which the derivative involves a combination of chain, product and quotient rules.
\end{itemize}

\vspace{16pt}

\noindent {\Large\bf{Need More Help?}}

\begin{itemize}\itemsep -2pt % Reduce space between items
\item  {\bf{Watch}} video \href{https://www.youtube.com/watch?v=8NveONuzUDY&feature=emb_title}{Inverse Functions (12:29)}. 

\item {\bf{Watch}} video \href{https://www.youtube.com/watch?v=yftDphiQVHg&feature=emb_title}{The Derivative of Inverse Functions (14:55)}. 
 
\item  {\bf{Watch}} video \href{https://www.youtube.com/watch?v=l_be36R3mKg&feature=emb_title}{The Derivative of the Natural Logarithm (9:39)}. 

\item {\bf{Do}} these problems.
\begin{enumerate}
\setcounter{enumi}{1}
\item 
\begin{enumerate}
\item For $x > 0$, find and simplify the derivative of $\displaystyle f(x) = \tan^{-1}x + \tan^{-1}(1/x)$.
%%\vspace{30mm}
\item What does your result tell you about $f$? 
%%\vspace{5mm}
\end{enumerate}

\item Use the graphs given to estimate the given \\
derivatives.   
\begin{enumerate}
\item $(f^{-1})'(5)$
\vspace{10mm}
\item $(f^{-1})'(15)$  
%%\item $h'(x)=1$
%%\item $h'(x)=-1$ 
\end{enumerate}

\vspace{-25mm}
\begin{figure}[H]
\begin{flushright}
	\includegraphics[width=3.5in]{fig_03_31.jpg}
	\end{flushright}
\end{figure} 

\item The figure shows $f(x)$ and $f^{-1}(x)$. More information is found in the table below.

\begin{flushleft}
\begin{tabular}{c|c|c}
\hline
$x$ & $f(x)$ & $f'(x)$ \\
\hline
0 & 1 & 0.7 \\
1 & 2 & 1.4 \\
2 & 4 & 2.8 \\
3 & 8 & 5.5 \\
\hline
\end{tabular}
\end{flushleft}

\vspace{-30mm}
\begin{figure}[H]
\begin{flushright}
	\includegraphics[width=2in]{fig_03_29.jpg}
	\end{flushright}
\end{figure}

\vspace{-40mm}
\begin{enumerate}
\item Find the values of $f(2)$, $f^{-1}(2)$, $f'(2)$, $(f^{-1})'(2)$.
\vspace{10mm}
\item Find the equation of the tangent lines at the points $P$ \\ and $Q$.  
\vspace{10mm}
\item What is the relationship between the two tangent lines?  
%%\vspace{20mm}
\end{enumerate}

\end{enumerate}
\end{itemize}

\vspace{16pt}

\noindent {\Large\bf{Selected Answers}}
\begin{enumerate}
\setcounter{enumi}{0}

\item 
\begin{enumerate}
\item $x$
\item $x$
\item $(-\infty,\infty)$; $(0,\infty)$
\item $(0,\infty)$; $(-\infty,\infty)$
\item $2^y$
\end{enumerate}

\item 
\begin{enumerate}
\item $f'(x) = 0$
\item $f(x)$ is constant
\end{enumerate}

\item 
\begin{enumerate}
\item $\displaystyle \frac{1}{f'(f^{-1}(5))} \approx \frac{1}{f'(15)} = \frac{1}{0.4} = 2.5$
\item $\displaystyle \frac{1}{f'(f^{-1}(15))} \approx \frac{1}{f'(30)} \approx \frac{1}{0.72}$
\end{enumerate}

\item 
\begin{enumerate}
\item $f(2) = 4$, $'(2) = 2.8$, $f^{-1}(2) = 1$, $(f^{-1})'(2) = \frac{5}{7}$
\item At $P = (3,8)$, the tangent line is $y=5.5(x-3)+8$.  At $Q=(8,3)$, the tangent line is $y=\frac{1}{5.5}(x-8)+3$.  
\item The tangent lines are inverses of each other (the graphs are reflections about the line $y=x$).
\end{enumerate}

\end{enumerate}

 

\end{document}

