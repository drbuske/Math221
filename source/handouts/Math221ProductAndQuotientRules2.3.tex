\documentclass[11pt,a4paper]{report}

\usepackage{amssymb,amsmath,epsfig,float,subfig,hyperref,multicol}

\usepackage{xcolor}

\definecolor{SCSUred}{HTML}{CD1041}

\hypersetup{colorlinks=true,linkcolor=SCSUred,urlcolor=SCSUred}

\usepackage{enumerate}
\usepackage{tikz}
\usetikzlibrary{arrows}
\usetikzlibrary{patterns}
\usetikzlibrary{decorations}
%%\usetikzlibrary{intersections}
\usetikzlibrary{matrix}
\usetikzlibrary{snakes}
\usetikzlibrary{calc}
\usetikzlibrary{backgrounds}

\definecolor{linecolor}{HTML}{0074C8}
\definecolor{linecolor2}{HTML}{C80200}

\newcommand{\imagebullet}[1]{\includegraphics[width=0.5cm]{#1}}


\pagestyle{empty}
\setlength{\textwidth}{7in}
\setlength{\textheight}{10in}
\setlength{\oddsidemargin}{-25pt}
\setlength{\evensidemargin}{-25pt}
\setlength{\topmargin}{-50pt}

\usepackage[english]{babel}
\usepackage[utf8]{inputenc}
\usepackage{fancyhdr}
 
%%\pagestyle{fancy}
\renewcommand{\headrulewidth}{0pt}
%%\fancyhf{}
%%\rhead{Share\LaTeX}
%%\lhead{Guides and tutorials}
%%\cfoot{OVER}


\newcommand{\DueA}{Tuesday, October 1}
\newcommand{\DueB}{Thursday, October 3}
\newcommand{\DueC}{Thursday, October 10}

\begin{document}

\begin{figure}[ht]
\begin{flushright}
	\includegraphics[width=2.0in]{U_PriHorz_WhtLtBG.jpg}
	\end{flushright}
\end{figure}

\vspace{-12mm}

\begin{flushleft}
\Large\bf \href{https://activecalculus.org/single/sec-2-3-prod-quot.html}{2.3 - The Product and Quotient Rules}\rm
%%Daily Preparation - \DueA \rm
\end{flushleft}


\vspace{8pt}

\noindent {\Large\bf{Overview}} \\
This section covers the following concepts: The {\it{product and quotient rules}} for differentiation.

%%This section covers the following concept: Linearization of a function.



\vspace{16pt}

%%\pagebreak

\noindent {\Large\bf{To prepare for class}} \\
Complete all actions listed below.  Respond to the questions highlighted with {\color{SCSUred}{\boxed{Submit}}}.  %% by the start of class on {\bf{\DueA}}.  A single .pdf should be uploaded to D2L Brightspace. 
\begin{itemize} \itemsep -2pt % Reduce space between items
\item {\bf{Read}} motivating questions and the introduction to \href{https://activecalculus.org/single/sec-2-3-prod-quot.html}{section 2.3} (up until Preview Activity 2.3.1).
\item[{\color{SCSUred} \boxed{Submit}}] {\bf{Do}} \href{https://activecalculus.org/single/sec-2-3-prod-quot.html#zLw}{Preview Activity 2.3.1} (via hand calculations).  

\item[{\color{SCSUred} \boxed{Submit}}]  {\bf{Check}} your answers to \href{https://activecalculus.org/single/sec-2-3-prod-quot.html#zLw}{Preview Activity 2.3.1} using \href{https://www.geogebra.org/classic?lang=en}{GeoGebra}.  Submit a screen capture of your work along with responses to the questions posed.  
\begin{itemize}
\item Enter the function $f(t) = 2t^2$ by typing {\tt{2*t\^{ }2}} into the first input box.  {\it{GeoGebra}} will define and label this function as $f(t)$ for you.  
\item Enter the function $g(t) = t^3+4t$ by typing {\tt{t\^{ }3 + 4*t}} into the second input box.  {\it{GeoGebra}} will define and label this function as $g(t)$ for you. 
\item In the third input box, type {\tt{f(t)*g(t)}}.  Then, relabel {\tt{h}} as {\tt{p}} (if you wish...just like in the activity).  [To relabel, just use backspace.]  
\item (Optional) If you like, you can simplify $p(t)$.  In this next input box, just type {\tt{Simplify(p(t))}}.  Nice, yes?  
\item {\it{GeoGebra}} has the ability to return derivatives of functions.  In the next input box, type {\tt{Derivative(p(t))}}.  {\it{GeoGebra}} will automatically label the result as $p'(t)$.  Does this match your hand-calculated result?  
\item In the next input box, we calculate $f'(t)\cdot g'(t)$ by typing {\tt{Derivative(f(t))*Derivative(g(t))}}.  Is the result the same as $p'(t)$?  [If needed, the {\tt{Simplify}} function (as described above) can be used.  
\item Do parts (d) and (e) of \href{https://activecalculus.org/single/sec-2-3-prod-quot.html#zLw}{Preview Activity 2.3.1} using {\it{GeoGebra}} to do the calculations.  
\end{itemize}

\item {\bf{Read}} \href{https://activecalculus.org/single/sec-2-3-prod-quot.html#MVQ}{section 2.3.1}.

 

\item {\bf{Watch}} video \href{https://www.youtube.com/watch?v=mkrnp3ew0WA&t=317s}{Product Rule Examples (6:47)}.  

\item[\imagebullet{CopilotLogo.jpg}] Prompt {\bf{Copilot}} ``I don't want a proof, but please explain in simple terms how the product rule in calculus comes about." 

\item[\imagebullet{CopilotLogo.jpg}] Prompt {\bf{Copilot}} ``Create an example of a typical product rule question in a college-level calculus class using a numerical table.  Be sure to provide me the solution too."  Is the solution provided correct? 

%%\item {\bf{Do}} Activity 2.3.2. 

\item {\bf{Read}} \href{https://activecalculus.org/single/sec-2-3-prod-quot.html#tcZ}{section 2.3.2}.
 
\item {\bf{Watch}} video \href{https://www.youtube.com/watch?v=bAGEnF0uFog}{Quick Review: The Product and Quotient Rules (1:59)}.

 

\item {\bf{Watch}} video \href{https://www.youtube.com/watch?v=HxFjkYjabwQ&feature=emb_title}{Quotient Rule Examples (10:37)}. 

\item[\imagebullet{CopilotLogo.jpg}] Prompt {\bf{Copilot}} ``I don't want a proof, but please explain in simple terms how the quotient rule in calculus comes about."  Is this helpful or unhelpful?



\item[{\color{SCSUred} \boxed{Submit}}] {\bf{Do}} \href{https://activecalculus.org/single/sec-2-3-prod-quot.html#tnQ}{Exercise 1 in section 2.3}.  Submit a screenshot to demonstrate completion.  

\item {\bf{Do}}  \href{https://activecalculus.org/single/sec-2-3-prod-quot.html#ApM}{Exercises 2-9 in section 2.3}.

 


\end{itemize}





\vspace{16pt}

\noindent {\Large\bf{After class}}\\
Solidifying the concepts discussed in class through practice is necessary to build your skills. 

%%\noindent {\large\bf{After \DueA}}
\begin{itemize}\itemsep -2pt % Reduce space between items

\item {\bf{Read}} \href{https://activecalculus.org/single/sec-2-3-prod-quot.html#Zki}{section 2.3.3}. 
\item {\bf{Watch}} video \href{https://www.youtube.com/watch?v=9lZNcY3VbdE&feature=emb_title}{Combining the Product and Quotient rules (9:00)}.  

%%\pagebreak

\item {\bf{Do}} this problem.

\begin{enumerate}
%%\item For $a$, $b$, $c$, and $d$ constant, find the derivative of $\displaystyle g(x) = \frac{ax+b}{cx+d}$.

\item Suppose the $f$ and $h$ are functions and that $f(3)=2$, $f'(3) = -2$, $h(3) = 1$, and $h'(3) = 4$.  
\begin{enumerate}
\item Calculate $m'(3)$, where $m(x) = f(x) \cdot h(x)$.  

\item Calculate $p'(3)$, where $\displaystyle p(x) = \frac{f(x)}{x^2h(x)}$. 

\end{enumerate}


\end{enumerate}

%%\item {\bf{Do}} Activity 2.3.3.  

%%\vspace{-8mm}
 
\item {\bf{Do}}  \href{https://activecalculus.org/single/sec-2-3-prod-quot.html#WAt}{Exercises 10-11 in section 2.3}. 
 


%%\end{itemize}

%%\noindent {\large\bf{After \DueB}}
%%\begin{itemize}\itemsep -2pt % Reduce space between items
%%\item {\bf{Finish}} any in-class activities you might not have finished during class.


\item {\bf{Read}} \href{https://activecalculus.org/single/sec-2-3-prod-quot.html#Frr}{section 2.3.4 (the summary)}. 

\item {\bf{Do}}  \href{https://activecalculus.org/single/sec-2-3-prod-quot.html#iOL}{Exercises 12-14 in section 2.3}.   
  
\item {\bf{Start working}} on the \href{https://www.myopenmath.com/index.php}{MOMwork} (MyOpenMath) assignment for this section.  %%This will be due on \DueC.  
\end{itemize}

%%\pagebreak

\vspace{16pt}

\noindent {\Huge\bf{Extra Prep}}

\vspace{16pt}

\noindent {\Large\bf{Basic learning objectives}}\\
These are the tasks you should be able to perform with reasonable
fluency when you arrive at our next class meeting. Important new
vocabulary words are indicated {\it{in italics}}.  Check each box when you feel confident you have a firm grasp on that objective.  

\begin{itemize} \itemsep -2pt % Reduce space between items
\renewcommand{\labelitemi}{\scriptsize$\square$}
\item Apply all the derivative rules from sections 2.1 and 2.2 with fluency.
\item Give a specific example to show that the derivative of a product of two functions, $f(x)\cdot g(x)$, is {\it{not}} the product of the derivatives, $f'(x)\cdot g'(x)$.

\item State and apply the product rule.

\item State and the quotient rule.
\end{itemize}

\vspace{16pt}

\noindent {\Large\bf{Advanced learning objectives}}\\
In addition to mastering the basic objectives, here are the tasks you should be able to perform after class, with practice:
\begin{itemize} \itemsep -2pt % Reduce space between items
\renewcommand{\labelitemi}{\scriptsize$\square$}
\item Prove the quotient rule using the product rule.

\item Differentiate a function for which the derivative involves a combination of the product rule, quotient rule, and other differentiation rules.

\item Use the product and quotient rules in the context of a real-world problem to find the slope of a tangent line, the instantaneous rate of change in a function, or the instantaneous velocity of an object.
\end{itemize}

\vspace{16pt}

\pagebreak

\noindent {\Large\bf{Need More Help?}}

\begin{itemize}\itemsep -2pt % Reduce space between items
\item {\bf{Watch}} \href{https://www.youtube.com/watch?time_continue=17&v=Jw3cZdDaKzo&feature=emb_title}{this introductory video} to help in your reading of section 2.3.1. 
\item {\bf{Watch}} video \href{https://www.youtube.com/watch?v=5L2qpUMh5c8&feature=emb_title}{computing derivatives using graphical information (6:49)}. 
\item {\bf{Do}} this problem.

\begin{enumerate}
%%\item For $a$, $b$, $c$, and $d$ constant, find the derivative of $\displaystyle g(x) = \frac{ax+b}{cx+d}$.
\setcounter{enumi}{1}
\item Use information from the graphs of $f(x)$ and $g(x)$ and the tangent lines shown to calculate each derivative.
\begin{enumerate}
\item $h'(2)$ if $h(x) = x^2 f(x)$
%%\vspace{40mm}
\item $k'(2)$ if $\displaystyle k(x) = \frac{f(x)}{g(x)}$
%%\vspace{40mm}
\item $m'(2)$ if $\displaystyle m(x) = \frac{x g(x)}{f(x)}$
\end{enumerate}

\vspace{-30mm}
 \begin{figure}[H]
\flushright
{
  \begin{tikzpicture}[thick,scale=0.75] [domain=-1:10]

\draw[color=blue, line width=1pt,domain=1:7]   plot (\x,{0.1*(\x-2)*(\x-5)*(\x-5)+3}) node [above] at (5,3) {$f(x)$};
\draw[color=black, line width=1pt,domain=1.25:4.2]   plot (\x,{0.9*\x+1.2});
\fill [black] (2,3) circle (4pt) node [left] at (2,3) {$(2,3)$};
\fill [black] (4,4.8) circle (4pt) node [left] at (4,4.8) {$(4,4.8)$};


\end{tikzpicture}}  
\end{figure}

\vspace{-5mm}
 \begin{figure}[H]
\flushright
{
  \begin{tikzpicture}[thick,scale=0.75] 

\draw[color=blue, line width=1pt,domain=-1:2]   plot (\x,{5-\x*\x}) node [above] at (-1.5,4) {$g(x)$};
\draw[color=black, line width=1pt,domain=0.7:2.1]   plot (\x,{-2*\x+6});
\draw[color=black, line width=1pt,domain=-2.25:1]   plot (\x,{5});
c
\fill [black] (1.5,3) circle (4pt) node [right] at (1.5,3) {$(3.5,3)$};
\fill [black] (0,5) circle (4pt) node [above] at (0,5) {$(2,5)$};
\fill [black] (-2,5) circle (4pt) node [above] at (-2,5) {$(0,5)$};

\end{tikzpicture}}  
\end{figure}


\end{enumerate}
\end{itemize}

%%\pagebreak

\vspace{16pt}
 
\noindent {\Large\bf{Selected Answers}}
\begin{enumerate}
\setcounter{enumi}{0}

\item  
\begin{enumerate}
\item $m'(x)=f'(x) \cdot h(x) + f(x) \cdot h'(x)$.  So, $m'(3) = f'(3) \cdot h(3) + f(3) \cdot h'(3) = (-2)(1) + (2)(4) = 6$
\item $\displaystyle p'(3)=-\frac{82}{27}$
\end{enumerate}

\item 
\begin{enumerate}
\item $h'(2)=15.6$
\item $k'(2)=0.18$
\item $\displaystyle m'(2)=\frac{2}{3}$
\end{enumerate}

\end{enumerate}

\end{document}

