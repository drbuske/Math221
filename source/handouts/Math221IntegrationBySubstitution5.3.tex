\documentclass[11pt,a4paper]{report}

\usepackage{amssymb,amsmath,epsfig,float,subfig,hyperref,multicol}
\usepackage{pgfplots}
\usepgfplotslibrary{fillbetween}
\usepackage{xcolor}

\definecolor{SCSUred}{HTML}{CD1041}

\hypersetup{colorlinks=true,linkcolor=SCSUred,urlcolor=SCSUred}

\usepackage{enumerate}
\usepackage{tikz}
\usetikzlibrary{arrows}
\usetikzlibrary{patterns}
\usetikzlibrary{decorations}
%%\usetikzlibrary{intersections}
\usetikzlibrary{matrix}
\usetikzlibrary{snakes}
\usetikzlibrary{calc}
\usetikzlibrary{backgrounds}

\definecolor{linecolor}{HTML}{0074C8}
\definecolor{linecolor2}{HTML}{C80200}

\newcommand{\imagebullet}[1]{\includegraphics[width=0.5cm]{#1}}

\pagestyle{empty}
\setlength{\textwidth}{7in}
\setlength{\textheight}{10in}
\setlength{\oddsidemargin}{-25pt}
\setlength{\evensidemargin}{-25pt}
\setlength{\topmargin}{-50pt}

\usepackage[english]{babel}
\usepackage[utf8]{inputenc}
\usepackage{fancyhdr}
 
%%\pagestyle{fancy}
\renewcommand{\headrulewidth}{0pt}
%%\fancyhf{}
%%\rhead{Share\LaTeX}
%%\lhead{Guides and tutorials}
%%\cfoot{OVER}

\newcommand{\DueA}{Monday, December 2}
\newcommand{\DueB}{Tuesday, December 3}
\newcommand{\DueC}{Monday, December 9}

\begin{document}

\begin{figure}[ht]
\begin{flushright}
	\includegraphics[width=2.0in]{U_PriHorz_WhtLtBG.jpg}
	\end{flushright}
\end{figure}

\vspace{-12mm}

\begin{flushleft}
\Large\bf \href{https://activecalculus.org/single/sec-5-3-substitution.html}{5.3 - Integration by Substitution}\rm
%%Daily Preparation - \DueA \rm
\end{flushleft}


\vspace{8pt}

\noindent {\Large\bf{Overview}} \\
In this section, we see how it is possible to reverse the chain rule. This technique is called integration by substitution. Knowing how to antidifferentiate basic functions and the substitution rule together will allow us to antidifferentiate more complex functions.


\vspace{16pt}

%%\pagebreak

\noindent {\Large\bf{To prepare for class}} \\
Complete all actions listed below.  Respond to the questions highlighted with {\color{SCSUred}{\boxed{Submit}}}.  %% by the start of class on {\bf{\DueA}}.  A single .pdf should be uploaded to D2L Brightspace. 
\begin{itemize} \itemsep -2pt % Reduce space between items

\item {\bf{Read}} motivating questions and the introduction to \href{https://activecalculus.org/single/sec-5-3-substitution.html}{section 5.3} (up until Preview Activity 5.3.1). 

\item[{\color{SCSUred} \boxed{Submit}}] {\bf{Do}} \href{https://activecalculus.org/single/sec-5-3-substitution.html#fEx}{Preview Activity 5.3.1} parts (a) and (b) numbers (i) and (ii). 
\begin{itemize}
\item (Optional) {\bf{Watch}} video \href{https://www.youtube.com/watch?v=2MLcQrIzQHk}{solution to Preview Activity 5.3.1 (7:23)}.  
\end{itemize} 

\item {\bf{Read}} \href{https://activecalculus.org/single/sec-5-3-substitution.html#cIr}{section 5.3.1} up to Activity 5.3.2.

%%\item {\bf{Watch}} \href{https://www.youtube.com/watch?v=EL48G_vtzKw&feature=emb_title}{Sketching the Graph of an Integral Function (8:15)}.

\item[\imagebullet{CopilotLogo.jpg}] Prompt {\bf{Copilot}} ``Walk me through an example in which $u$-substitution is used to calculate an indefinite integral."

\item {\bf{Read}} \href{https://activecalculus.org/single/sec-5-3-substitution.html#IPA}{section 5.3.2}.  

\item {\bf{Watch}} video \href{https://www.youtube.com/watch?v=p_z5AU2z4DI&feature=emb_title}{Integration by Substitution Example 1 (3:43)}.  

\item {\bf{Watch}} video \href{https://www.youtube.com/watch?v=G7zJdIpXIq8&feature=emb_title}{Integration by Substitution Example 2 (3:50)}. 

\item {\bf{Watch}} video \href{https://www.youtube.com/watch?v=_Y-01MZOebQ&feature=emb_title}{Integration by Substitution Example 3 (4:42)}. 

\item {\bf{Watch}} video \href{https://www.youtube.com/watch?v=NjcjFwX653s&feature=emb_title}{Integration by Substitution Example 4 (4:19)}. 

\item {\bf{Watch}} video \href{https://www.youtube.com/watch?v=YPle_z6tOeY&feature=emb_title}{Integration by Substitution Example 5 (3:40)}. 

\item {\bf{Do}} these problems.
\begin{enumerate}
\setcounter{enumi}{1}

\item[{\color{SCSUred} \boxed{Submit}} 1.]  {\bf{Explore}} the applet \href{https://www.geogebra.org/m/YPSNUfpm}{U-Substitition by Tim Brzezinski}. It illustrates the idea of $u$-substitition.  No matter the value of $a$, $b$, of how you position the slider, how are the areas of the green and blue regions related?  

\item 
\begin{enumerate}
\item Using the $u$-substitution $u(x) = x^3$, find $\displaystyle \int 3x^2 \cos(x^3) \ dx$. 
\item Repeat (a) using the applet \href{https://www.geogebra.org/m/uzSwyMa5}{Integration by Substitution by Ravinder Kumar}.  Note that you will need to enter the integrand {\it{and}} the function $u(x)$.  It then gives step by step instructions.  Note also that the applet uses $v(x)$ rather than $u(x)$.  Does your answer match that you found in (a)?  
\item Use the applet to check the correctness of \href{https://activecalculus.org/single/sec-5-3-substitution.html#tGC}{Example 5.3.2 in the text}.  
\end{enumerate}



\end{enumerate}
 
  

 
\item {\bf{Watch}} video \href{https://www.youtube.com/watch?v=uTT6e-yXZyA&feature=emb_title}{Quick Recap - Integration by Substitution (3:27)}.

\item[{\color{SCSUred} \boxed{Submit}} ] {\bf{Do}} write a paragraph or two with your reflections on the value of each of the following as they appeared in the Daily Prep assignments this semester.  These comments will be used to improve the Daily Prep assignments for the next semester.  
\begin{enumerate}
\item The Readings from the Textbook
\item The Interactive Applets (often using {\it{GeoGebra}})
\item The Videos
\item The Problems in the textbook (interactive and written)
\item The Problems not in the textbook
\end{enumerate}  


\end{itemize}













\vspace{16pt}

\noindent {\Large\bf{After class}}\\
Solidifying the concepts discussed in class through practice is necessary to build your skills. 

%%\noindent {\large\bf{After \DueA}}
\begin{itemize}\itemsep -2pt % Reduce space between items



 

\item {\bf{Read}} \href{https://activecalculus.org/single/sec-5-3-substitution.html#oWJ}{section 5.3.3}.

\item {\bf{Explore}} the applet \href{https://www.geogebra.org/m/NKhaCFhH}{Integration by Substitution by John Golden}.  Answer the single question of `Why are these definite integrals equal?'  In fact, write down both integrands and both limits of integration and then approximate each definite integral using a Riemann Sum via the applet \href{https://www.geogebra.org/m/RCVce5W4}{Riemann Sums by J Mulholland}.  This should convince you that both definite integrals do indeed have the same value.  

\item {\bf{Watch}} video \href{https://www.youtube.com/watch?v=Iia2ZdvYFIc&feature=emb_title}{Integration by Substitution Example 6 (3:08)}. 

\item {\bf{Watch}} video \href{https://www.youtube.com/watch?v=buu_jlQBAjI&feature=emb_title}{Definite Integrals using Substitution (4:31)}. 

\item[\imagebullet{CopilotLogo.jpg}] Prompt {\bf{Copilot}} ``If the function $u$ is not one-to-one, the method of $u$-substitution as applied to a definite integral may fail.  Give me an example showing this."

\item {\bf{Do}} \href{https://activecalculus.org/single/sec-5-3-substitution.html#lDN}{exercises 1-3 in section 5.3}.

\item {\bf{Do}} these problems.
\begin{enumerate}
\setcounter{enumi}{2}
\item Find $\displaystyle \int x^3 \sqrt{x^4+5} \ dx$ by letting $u(x) = x^4 + 5$.  
\item Find $\displaystyle \int x\sqrt{x+2} \ dx$.  
\item Find $\displaystyle \int \tan(x) \ dx$.  
\item Find $\displaystyle \int \sec(x) \ dx$.  {\it{Hint:}} Multiply the integrand by $\displaystyle \frac{\sec(x) + \tan(x)}{\sec(x) + \tan(x)}$.  


\end{enumerate}
 

%%\item {\bf{Do}} \href{https://activecalculus.org/single/sec-5-2-FTC2.html#xaU}{exercise 4 in section 5.2}.



%%\end{itemize}

%%\noindent {\large\bf{After \DueB}}
%%\begin{itemize}\itemsep -2pt % Reduce space between items


  
\item {\bf{Read}} \href{https://activecalculus.org/single/sec-5-3-substitution.html#VdS}{section 5.3.4 - summary}. 



\item {\bf{Do}} \href{https://activecalculus.org/single/sec-5-3-substitution.html#dwy}{exercises 7-10 in section 5.3}.

\item {\bf{Start working}} on the \href{https://www.myopenmath.com/index.php}{MOMwork} (MyOpenMath) assignment for this section.  %%This will be due on \DueC. 

\end{itemize}


%%\pagebreak

\vspace{16pt}

\noindent {\Huge\bf{Extra Prep}}


\vspace{16pt}

\noindent {\Large\bf{Basic learning objectives}}\\
These are the tasks you should be able to perform with reasonable
fluency when you arrive at our next class meeting. Important new
vocabulary words are indicated {\it{in italics}}.   Check each box when you feel confident you have a firm grasp on that objective.  

\begin{itemize} \itemsep -2pt % Reduce space between items
\renewcommand{\labelitemi}{\scriptsize$\square$}
\item Recognize the notation for an {\it{indefinite integral}} and state its meaning.
\item Determine the general antiderivative given a composite function whose ``inner'' function is linear.
\end{itemize}

\vspace{16pt}

\noindent {\Large\bf{Advanced learning objectives}}\\
In addition to mastering the basic objectives, here are the tasks you should be able to perform after class, with practice:
\begin{itemize} \itemsep -2pt % Reduce space between items
\renewcommand{\labelitemi}{\scriptsize$\square$} 
\item Use proper notation when using Integration by Substitution. In particular, fully convert an integral from $x$'s to $u$'s and back again, without mixing the two variables.
\item Use Integration by Substitution in unusual cases, such as those where there is not an obvious substitution.
\end{itemize}

\vspace{16pt}

\noindent {\Large\bf{Need More Help?}}

\begin{itemize}\itemsep -2pt % Reduce space between items

\item  {\bf{Watch}} video \href{https://www.youtube.com/watch?v=pBfyWU_lf04&feature=youtu.be}{U-Substitution for Antiderivatives (8:13)}.  

\item {\bf{Finish (if needed)}} \href{https://activecalculus.org/single/sec-5-3-substitution.html#YRF}{Activity 5.3.2}.  
\begin{itemize}
\item  {\bf{Watch}} video \href{https://www.youtube.com/watch?v=bqot6L-41Ag}{solution to (a)-(c) of Activity 5.3.2 (4:08)} and \href{https://www.youtube.com/watch?v=4S4LxP62eqs}{solution to (d)-(f) of Activity 5.3.2 (4:57)}. 
\end{itemize}

\item {\bf{Finish (if needed)}} \href{https://activecalculus.org/single/sec-5-3-substitution.html#bqM}{Activity 5.3.3}.  
\begin{itemize}
\item (Optional) {\bf{Watch}} video \href{https://www.youtube.com/watch?v=CN0F8OGkLhU}{solution to Activity 5.3.3 (4:36)}.
\end{itemize} 

\item  {\bf{Do}} more practice via the applet \href{https://www.geogebra.org/m/y95xjghz}{Integration by Substitution by R. Kumar}.  Clicking `Practice' generates a new problem.  Clicking `Show Answer' can be used to check your work.  
\end{itemize}

\vspace{16pt}

\noindent {\Large\bf{Selected Answers}}
\begin{enumerate}
\setcounter{enumi}{0}

\item The blue and green regions have exactly the same area.  

\item 
\begin{enumerate}
\item $\sin(x^3) + C$
\item Yes, it matches.
\end{enumerate}

\item $\displaystyle \int x^3 \sqrt{x^4+5} \ dx$ by letting $u(x) = x^4 + 5 = \frac{1}{6}(x^4+5)^{3/2} + C$

\item Let $u(x) = x+2$.  Then $x = u-2$.  This allows one to write the integral in terms of $u$ rather easily.  Thus, $\displaystyle \int x\sqrt{x+2} \ dx = \frac{2}{5}(x+2)^{5/2} - \frac{4}{3}(x+2)^{3/2}+C$

\item Use the substitution $u=\cos x$.  Then $\displaystyle \int \tan(x) \ dx = \int \frac{\sin x}{\cos x} \ dx = -\int \frac{1}{u} \ du = -\ln |\cos x| + C$.

\item $\displaystyle \int \sec(x) \ dx = \ln | \sec x + \tan x | + C$
\end{enumerate}

\end{document}















