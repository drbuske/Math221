\documentclass[11pt,a4paper]{report}

\usepackage{amssymb,amsmath,epsfig,float,subfig,hyperref,multicol}

\usepackage{xcolor}

\definecolor{SCSUred}{HTML}{CD1041}

\hypersetup{colorlinks=true,linkcolor=SCSUred,urlcolor=SCSUred}

\usepackage{enumerate}
\usepackage{tikz}
\usetikzlibrary{arrows}
\usetikzlibrary{patterns}
\usetikzlibrary{decorations}
%%\usetikzlibrary{intersections}
\usetikzlibrary{matrix}
\usetikzlibrary{snakes}
\usetikzlibrary{calc}
\usetikzlibrary{backgrounds}

\definecolor{linecolor}{HTML}{0074C8}
\definecolor{linecolor2}{HTML}{C80200}

\newcommand{\imagebullet}[1]{\includegraphics[width=0.5cm]{#1}}


\pagestyle{empty}
\setlength{\textwidth}{7in}
\setlength{\textheight}{10in}
\setlength{\oddsidemargin}{-25pt}
\setlength{\evensidemargin}{-25pt}
\setlength{\topmargin}{-50pt}

\usepackage[english]{babel}
\usepackage[utf8]{inputenc}
\usepackage{fancyhdr}
 
%%\pagestyle{fancy}
\renewcommand{\headrulewidth}{0pt}
%%\fancyhf{}
%%\rhead{Share\LaTeX}
%%\lhead{Guides and tutorials}
%%\cfoot{OVER}

\newcommand{\DueA}{Monday, October 14}
\newcommand{\DueB}{Tuesday, October 15}
\newcommand{\DueC}{Friday, October 18}

\begin{document}

%% Math 327 1.1

%%\begin{flushright}
%%Name:\underline{\hspace{50mm}}
%%\end{flushright}

%%\vspace{-20mm}


\begin{figure}[ht]
\begin{flushright}
	\includegraphics[width=2.0in]{U_PriHorz_WhtLtBG.jpg}
	\end{flushright}
\end{figure}

\vspace{-12mm}

\begin{flushleft}
\Large\bf \href{https://activecalculus.org/single/sec-2-7-implicit.html}{2.7 - Derivatives of Functions Given Implicitly} \rm
%%Daily Preparation - \DueA \rm
\end{flushleft}


\vspace{8pt}

\noindent {\Large\bf{Overview}} \\
In several different contexts, we will see that it is useful to consider a function that is described implicitly, rather than explicitly. A prominent simple example is that of a circle: while no single function of the form $y=f(x)$ can represent every point on the circle, there is nonetheless a key relationship between the $x$ and $y$ coordinates of points on the curve. In more complicated settings, we'll see that while there is a relationship between $x$ and $y$, there is no way to explicitly solve for $y$ in terms of $x$. In this setting and others similar to it, we want to be able to still compute $\displaystyle \frac{dy}{dx}$. The process of implicit differentiation enables us to do so.





\vspace{16pt}

%%\pagebreak

\noindent {\Large\bf{To prepare for class}} \\
Complete all actions listed below.  Respond to the questions highlighted with {\color{SCSUred}{\boxed{Submit}}}.  %% by the start of class on {\bf{\DueA}}.  A single .pdf should be uploaded to D2L Brightspace.
\begin{itemize} \itemsep -2pt % Reduce space between items


\item {\bf{Read}} motivating questions and the introduction to \href{https://activecalculus.org/single/sec-2-7-implicit.html}{section 2.7} (up until Preview Activity 2.7.1).

\item[\imagebullet{CopilotLogo.jpg}] Prompt {\bf{Copilot}} ``What does it mean to say that a curve is an implicit function of $x$, rather than an explicit function of $x$?"

\item[{\color{SCSUred} \boxed{Submit}}]  {\bf{Do}} \href{https://activecalculus.org/single/sec-2-7-implicit.html#zpX}{Preview Activity 2.7.1}.
\begin{itemize}
\item (Optional) {\bf{Watch}} video \href{https://www.youtube.com/watch?v=DSP3rbdX2F8}{solution to Preview Activity 2.7.1 (4:17)}
\end{itemize}
\item[{\color{SCSUred} \boxed{Submit}}]  {\bf{Do}} the following computation in \href{https://www.geogebra.org/classic}{GeoGebra}.  Submit screenshots as needed.
\begin{itemize}
\item Enter {\tt{sin(x)}} in the first input box.  \href{https://www.geogebra.org/classic}{GeoGebra} will assign this as the function $f(x)$.  
\item Enter {\tt{Derivative($x^2\cdot f(x)$)}} in the next input box.  Does the result match what you expected based on your answer to (b) in \href{https://activecalculus.org/single/sec-2-7-implicit.html#zpX}{Preview Activity 2.7.1}?  
\item Change the function defined in the first input box from {\tt{sin(x)}} to {\tt{$2^x$}}.  Again, does the result match what you expected based on your answer to (b) in \href{https://activecalculus.org/single/sec-2-7-implicit.html#zpX}{Preview Activity 2.7.1}?  
\end{itemize}

\item {\bf{Read}} \href{https://activecalculus.org/single/sec-2-7-implicit.html#QWc}{section 2.7.1}.  This is a bit longer than most sections you have read to this point.  

\item {\bf{Watch}} video \href{https://www.youtube.com/watch?v=YI7uxdvcq4E&list=PL9bIjQJDwfGuXQHuS5Jkmum_CFILoCZX-&index=49}{Quick Review - Derivatives of Functions Given Implicitly (2:22)}.  

\item[\imagebullet{CopilotLogo.jpg}] Prompt {\bf{Copilot}} ``Walk me through an example of computing $dy/dx$ for an equation in which $y$ is defined by $x$ implicitly.  The graph of the equation should have 2 or more slopes at the $x$-value chosen to compute $dy/dx$ at."

\item[{\color{SCSUred} \boxed{Submit}}]  {\bf{Do}} write a short explanation of the difference between writing $\frac{d}{dx}[x^2 + y^2]$ and $\frac{dy}{dx}[x^2 + y^2]$.  That is, explain the big difference between the meanings of the symbols $\frac{d}{dx}$ and $\frac{dy}{dx}$.  

\item {\bf{Watch}} video \href{https://www.youtube.com/watch?v=MEaZ44dkiuE&feature=youtu.be}{Implicit Differentiation - Part 1 - The Circle (4:18)}.

\item {\bf{Watch}} video \href{https://www.youtube.com/watch?v=FeQlUqXqa6Y&feature=youtu.be}{Implicit Differentiation - Part 2 - The Cardioid (3:48)}. 

\item {\bf{Watch}} video \href{https://www.youtube.com/watch?v=wEiiLU2jFng&feature=emb_title}{Derivatives of Implicit Functions (5:05)}.

\item {\bf{Watch}} video: \href{https://www.youtube.com/watch?v=_2aCDXYMz1U&list=PL9bIjQJDwfGuXQHuS5Jkmum_CFILoCZX-&index=51}{Finding Slope with Implicit Differentiation (8:20)}.  




\end{itemize}









\vspace{16pt}

\noindent {\Large\bf{After class}}\\
Solidifying the concepts discussed in class through practice is necessary to build your skills. 

%%\noindent {\large\bf{After \DueA}}
\begin{itemize}\itemsep -2pt % Reduce space between items
\item {\bf{Read}} \href{https://activecalculus.org/single/sec-2-7-implicit.html#xdl}{section 2.7.2 (summary)}.


\item {\bf{Do}}  \href{https://activecalculus.org/single/sec-2-7-implicit.html#amx}{Exercises 1-3 in section 2.7.3}.

\pagebreak
 
\item {\bf{Do}} the following problem.
\begin{enumerate}
\item Find the equation of the tangent line to the curve $x^3+x^2y+2y^2=2$ at the point $(1,0.5)$.  Your tangent line should match that shown on the diagram.    
\begin{figure}[H]
\begin{flushright}
	\includegraphics[width=2.25in]{ImplicitGraph1.jpg}
\end{flushright}
\end{figure}

\end{enumerate}

%%\item {\bf{Do}} Activity 2.7.3. 

\item {\bf{Explore}} an applet \href{http://webspace.ship.edu/msrenault/GeoGebraCalculus/derivative_implicit.html}{Implicit Differentiation} that will make you the world champion of understanding how to use implicit differentiation on circles, ellipses, and hyperbolae.  As always, do the three `explore' questions to fully benefit from the activity.  If you can survive the challenge, you probably no longer need to study this topic!   

%%\end{itemize}

%%\noindent {\large\bf{After \DueB}}
%%\begin{itemize}\itemsep -2pt % Reduce space between items 
 
\item {\bf{Do}} \href{https://activecalculus.org/single/sec-2-7-implicit.html#VYd}{Exercises 4-5 in section 2.7.3}.

\item {\bf{Do}}  \href{https://activecalculus.org/single/sec-2-7-implicit.html#imv}{Exercises 6-8 in section 2.7.3}.  Use  \href{https://www.geogebra.org/classic}{GeoGebra} to check the reasonableness of your solutions to exercises 6 and 7.  

\item {\bf{Start working}} on the \href{https://www.myopenmath.com/index.php}{MOMwork} (MyOpenMath) assignment for this section.  %%This will be due on \DueC. 

\end{itemize}

 
 
\pagebreak

\vspace{16pt}

\noindent {\Huge\bf{Extra Prep}}

\vspace{16pt}

\noindent {\Large\bf{Basic learning objectives}}\\
These are the tasks you should be able to perform with reasonable
fluency when you arrive at our next class meeting. Important new
vocabulary words are indicated {\it{in italics}}.  Check each box when you feel confident you have a firm grasp on that objective. 

\begin{itemize} \itemsep -2pt % Reduce space between items
\renewcommand{\labelitemi}{\scriptsize$\square$}
\item (Review) Given a basic differentiation rule, gives its chain rule version.  For example, the chain rule version of $\frac{d}{dx}[x^n] = nx^{n-1}$ is:
$$\frac{d}{dx}[y^n] \cdot \frac{dy}{dx} \ \ \ {\rm{or}} \ \ \ \frac{d}{dx}[y^n] = ny^{n-1} \cdot y'$$
\item State an example of a function that is defined explicitly by a relation between $x$ and $y$, and give an example where the relation is implicit.
\item Determine if a given point lies on the graph of the equation in $x$ and $y$, where $y$ is an implicit function of $x$.

\item Implicitly differentiate basic expressions such as $\frac{d}{dx}[x^2f(x)]$, where $f$ does not have a known formula.

\item Recognize the difference between the notations ``$\frac{d}{dx}[ \ \ ]$" and ``$\frac{dy}{dx}$."

\item Use the different notations $\frac{dy}{dx}$ and $\frac{dy}{dx}\big|_{(a,b)}$ appropriately, given a certain context.
\end{itemize}

\vspace{16pt}

\noindent {\Large\bf{Advanced learning objectives}}\\
In addition to mastering the basic objectives, here are the tasks you should be able to perform after class, with practice:
\begin{itemize} \itemsep -2pt % Reduce space between items
\renewcommand{\labelitemi}{\scriptsize$\square$}
\item Determine $\frac{dy}{dx}$ via implicit differentiation for complicated curves such as $x^3+6xy+y^3=1$ and $\sin(y)+y=x^3+x$.

\item Find the slope and equation of a tangent line to a curve specified by an equation that is not the graph of a function.

\item Given an implicitly defined curve and a point on the curve, find the local linearization at this point to approximate the coordinates of nearby points.

\item Understand how to use $\frac{dy}{dx}$ (which may depend on both $x$ and $y$) to determine all points where the tangent line to a given implicit curve is horizontal or vertical.
\end{itemize}

\vspace{16pt}

\noindent {\Large\bf{Need More Help?}}

\begin{itemize}\itemsep -2pt % Reduce space between items

\item  {\bf{Watch}} video \href{https://www.youtube.com/watch?v=n-Kl3atgYbk&feature=youtu.be}{Implicit Differentiation: Student Problem Solving (4:05)}.  It shows how impossible it can be to solve an equation explicitly!
\item  {\bf{Watch}} this video which includes detailed explanation of \href{https://activecalculus.org/single/sec-2-7-implicit.html#dku}{Example 2.7.3 in the text}: \href{https://www.youtube.com/watch?v=9YZFD8lou_0&feature=emb_title}{Implicit Differentiation (12:37)}.

\item {\bf{Watch}} this video \href{https://www.youtube.com/watch?v=w13MzXWVVUw&list=PLJtEcQL1-E8V9Fs1E1-KY9nyL9y2JHh8-&index=42}{Implicit and Explicit Functions (9:04)} for even more examples of this process.  

\item {\bf{Watch}} this video \href{https://www.youtube.com/watch?v=lVEbRfkgmU0&list=PLJtEcQL1-E8V9Fs1E1-KY9nyL9y2JHh8-&index=41}{Using Implicit Differentiation (11:08)} to continue seeing examples.  


\item {\bf{Watch}} video: \href{https://www.youtube.com/watch?v=B_mZpC6dFdA&feature=emb_title}{Implicit Differentiation (5:42)}.  
\item {\bf{Watch}} video \href{https://www.youtube.com/watch?v=fIy5Wav4rok&feature=emb_title}{How to Do Implicit Differentiation (14:16)}. 
\end{itemize}
 
 \vspace{16pt}

\noindent {\Large\bf{Selected Answers}}
\begin{enumerate}
\setcounter{enumi}{0}

\item $\displaystyle y=-\frac{4}{3}x + \frac{11}{6}$


\end{enumerate}



\end{document}

