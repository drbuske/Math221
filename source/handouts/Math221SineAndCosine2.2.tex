\documentclass[11pt,a4paper]{report}

\usepackage{amssymb,amsmath,epsfig,float,subfig,hyperref,multicol}

\usepackage{xcolor}

\definecolor{SCSUred}{HTML}{CD1041}

\hypersetup{colorlinks=true,linkcolor=SCSUred,urlcolor=SCSUred}

\usepackage{enumerate}
\usepackage{tikz}
\usetikzlibrary{arrows}
\usetikzlibrary{patterns}
\usetikzlibrary{decorations}
%%\usetikzlibrary{intersections}
\usetikzlibrary{matrix}
\usetikzlibrary{snakes}
\usetikzlibrary{calc}
\usetikzlibrary{backgrounds}

\definecolor{linecolor}{HTML}{0074C8}
\definecolor{linecolor2}{HTML}{C80200}

\newcommand{\imagebullet}[1]{\includegraphics[width=0.5cm]{#1}}

\pagestyle{empty}
\setlength{\textwidth}{7in}
\setlength{\textheight}{10in}
\setlength{\oddsidemargin}{-25pt}
\setlength{\evensidemargin}{-25pt}
\setlength{\topmargin}{-50pt}

\usepackage[english]{babel}
\usepackage[utf8]{inputenc}
\usepackage{fancyhdr}
 
%%\pagestyle{fancy}
\renewcommand{\headrulewidth}{0pt}
%%\fancyhf{}
%%\rhead{Share\LaTeX}
%%\lhead{Guides and tutorials}
%%\cfoot{OVER}

\newcommand{\DueA}{Monday, September 30}
\newcommand{\DueB}{Monday, September 30}
\newcommand{\DueC}{Monday, October 7}

\begin{document}


\begin{figure}[ht]
\begin{flushright}
	\includegraphics[width=2.0in]{U_PriHorz_WhtLtBG.jpg}
	\end{flushright}
\end{figure}

\vspace{-12mm}

\begin{flushleft}
\Large\bf \href{https://activecalculus.org/single/sec-2-2-sin-cos.html}{2.2 - The Sine and Cosine Functions}\rm
%% Daily Preparation - \DueA \rm
\end{flushleft}


\vspace{8pt}

\noindent {\Large\bf{Overview}} \\
This section covers the following concepts: {\it{Better understanding of the derivative of an exponential function, derivatives of the sine and cosine functions.}}

%%This section covers the following concept: Linearization of a function.



\vspace{16pt}

%%\pagebreak

\noindent {\Large\bf{To prepare for class}} \\
Complete all actions listed below.  Respond to the questions highlighted with {\color{SCSUred}{\boxed{Submit}}}.  %% by the start of class on {\bf{\DueA}}.  A single .pdf should be uploaded to D2L Brightspace.
\begin{itemize} \itemsep -2pt % Reduce space between items


\item {\bf{Read}} the motivating questions and the introduction to \href{https://activecalculus.org/single/sec-2-2-sin-cos.html}{section 2.2} (up until Preview Activity 2.2.1).

\item[{\color{SCSUred} \boxed{Submit}}] {\bf{Do}} the following construction in \href{https://www.geogebra.org/classic?lang=en}{\it{GeoGebra}}.  Submit screenshots and responses to the questions posed. 
\begin{itemize}
\item In the first input box, type {\tt{sin(x)}}.  This will define and graph $f(x)=\sin(x)$. 
\item To get a visual {\it{estimate}} of the derivative, we will use the definition of derivative which tells us that $$f'(x) = \lim_{h \rightarrow 0} \frac{f(x+h)-f(x)}{h}.$$
Rather than take a limit, we will use a reasonably small value of $h$, say $h = 0.01$.  Then, $$f'(x) \approx \frac{f(x+0.01)-f(x)}{0.01}.$$
In the next input box, type {\tt{(f(x+0.01)-f(x))/0.01}}.  This will define and graph $$\displaystyle g(x) =\frac{f(x+0.01)-f(x)}{0.01} \approx f'(x) = \frac{d}{dx}\sin x.$$  Have you seen the graph of $g(x)$ before?  Identify it and fill the following blank in accordingly:
$$\frac{d}{dx} \sin x = \underline{\hspace{30mm}}$$
\item {\it{Repeat}} this whole process to visualize the derivative (at least an estimate of it) of $f(x) = \cos x$.  
\end{itemize}


  
\item {\bf{Read}} \href{https://activecalculus.org/single/sec-2-2-sin-cos.html#QVP}{section 2.2.1}. 

%%\item {\bf{Do}} Activity 2.2.4.  

\item[{\color{SCSUred} \boxed{Submit}}] {\bf{Do}} some exploration with {\it{GeoGebra}} to conjecture the value of the following limits:
\begin{itemize}
\renewcommand{\labelitemii}{\scriptsize$\blacksquare$}
\item $\displaystyle \lim_{h \rightarrow 0} \frac{\cos h - 1}{h}$
\item $\displaystyle \lim_{h \rightarrow 0} \frac{\sin h}{h}$
\end{itemize}
One suggestion is to plot each of these expressions of $h$ (namely $\frac{\cos h - 1}{h}$ and $\frac{\sin h}{h}$ - the behavior of each graph near $h=0$ should indicate the value of each limit.  [Important note: {\it{GeoGebra}} will not treat $h$ as a variable.  So, use $x$ rather than $h$ in order to explor each limit.  That is, investigate $\displaystyle \lim_{x \rightarrow 0} \frac{\cos x - 1}{x}$ rather than $\displaystyle \lim_{h \rightarrow 0} \frac{\cos h - 1}{h}$.] Submission of screenshots would be ideal.  

\item[\imagebullet{CopilotLogo.jpg}] Prompt {\bf{Copilot}} ``Why is the derivative of $\sin(x)$ only $\cos(x)$ when $x$ is measured in radians and not degrees?"

\item[\imagebullet{CopilotLogo.jpg} {\color{SCSUred} \boxed{Submit}} ] First, try to answer this question before prompting the AI.  Then, prompt {\bf{Copilot}} ``Can you give me an example of a function whose derivative is periodic but it itself is not periodic?"  Is the response reasonable? 


\end{itemize}










\vspace{16pt}

\noindent {\Large\bf{After class}}\\
Solidifying the concepts discussed in class through practice is necessary to build your skills. 

%%\noindent {\large\bf{After \DueA}}
\begin{itemize}\itemsep -2pt % Reduce space between items 
%%\item {\bf{Finish}} any in-class activities you might not have finished during class.
\item {\bf{Read}} \href{https://activecalculus.org/single/sec-2-2-sin-cos.html#xcY}{section 2.2.2 (the summary)}.
\item {\bf{Do}}  \href{https://activecalculus.org/single/sec-2-2-sin-cos.html#gZy}{Exercises 1-3 in section 2.2}.
  
\item {\bf{Start working}} on the \href{https://www.myopenmath.com/index.php}{MOMwork} (MyOpenMath) assignment for this section.  %%This will be due on \DueC.  
 
\end{itemize}

%%\pagebreak

\vspace{16pt}

\noindent {\Huge\bf{Extra Prep}}

\vspace{16pt}

\noindent {\Large\bf{Basic learning objectives}}\\
These are the tasks you should be able to perform with reasonable
fluency when you arrive at our next class meeting. Important new
vocabulary words are indicated {\it{in italics}}.  Check each box when you feel confident you have a firm grasp on that objective.  

\begin{itemize} \itemsep -2pt % Reduce space between items
\renewcommand{\labelitemi}{\scriptsize$\square$}
\item Identify the graph of a general exponential function $f(x)=a^x$ and state its most important features (e.g. $y$-intercept, behavior for large positive or negative $x$-values).  (See \href{http://mathquest.carroll.edu/CarrollActiveCalculus/S_0_2_Exponentials.html}{this website} if review is needed.)
\item State the general rule for $\displaystyle \frac{d}{dx}a^x$ (where $a>0$ is real).  
\item Identify what makes the function $e^x$ special in terms of its relationship to its own derivative.

\item State the values of $\sin x$ and $\cos x$ at the angle values $0$, $\pi/6$, $\pi/4$, $\pi/3$, $\pi/2$ and other related points on the unit circle without a calculator. (See \href{http://mathquest.carroll.edu/CarrollActiveCalculus/S_0_5_TrigFunctions.html}{this website} for review if needed.)

\item State the derivatives of the functions $y=\sin x$ and $y= \cos x$.
\end{itemize}

\vspace{16pt}

\noindent {\Large\bf{Advanced learning objectives}}\\
In addition to mastering the basic objectives, here are the tasks you should be able to perform after class, with practice:
\begin{itemize} \itemsep -2pt % Reduce space between items
\renewcommand{\labelitemi}{\scriptsize$\square$}
\item Use the derivatives of $y=\sin(x)$ and $y=\cos(x)$ in the context of a real-world problem to find the slope of a tangent line, the instantaneous rate of change in a function, or the instantaneous velocity of an object.
\item Prove that $\displaystyle \frac{d}{dx}\sin x=\cos x$.
\end{itemize}

\vspace{16pt}

\noindent {\Large\bf{Need More Help?}}

\begin{itemize}\itemsep -2pt % Reduce space between items
\item If you are not yet convinced, {\bf{explore}} the applet \href{https://www.geogebra.org/m/qwdxbtGF}{https://www.geogebra.org/m/qwdxbtGF} can be used to help you believe the formula found in the above activity.  Just start with the function $\sin x$ or $\cos x$ accordingly.

\item {\bf{Watch}} the video: \href{https://www.youtube.com/watch?v=CJuLeDHSBEU&feature=emb_title}{On computing exact trigonometric values at key angles (3:56)}.  This could be especially helpful as a refresher on remembering basic facts an easier way.  

\item {\bf{Watch}} the video: \href{https://www.youtube.com/watch?v=CJuLeDHSBEU&feature=emb_title}{How to remember trig values on the unit circle (12:02)}.  
\end{itemize}
 

\end{document}















\noindent {\large\bf{After \DueA}}
\begin{itemize}\itemsep -2pt % Reduce space between items
\item {\bf{Finish}} any in-class activities you might not have finished during class.
\item {\bf{Do}} Exercise 3 in \href{https://activecalculus.org/single/sec-2-2-sin-cos.html}{section 2.2}. 
  
\item {\bf{Start working}} on the \href{https://www.myopenmath.com/index.php}{MOMwork} (MyOpenMath) assignment for this section.  This will be due on Thursday, October 1.  
\end{itemize}

